% Created 2016-01-22 Fri 09:22
\documentclass[letter,twoside,twocolumn]{article}
\usepackage[utf8]{inputenc}
\usepackage[T1]{fontenc}
\usepackage{fixltx2e}
\usepackage{graphicx}
\usepackage{longtable}
\usepackage{float}
\usepackage{wrapfig}
\usepackage{rotating}
\usepackage[normalem]{ulem}
\usepackage{amsmath}
\usepackage{textcomp}
\usepackage{marvosym}
\usepackage{wasysym}
\usepackage{amssymb}
\usepackage{hyperref}
\tolerance=1000
\author{Ryan Sharif}
\date{\today}
\title{Week 3 Discussion: Harmony}
\hypersetup{
  pdfkeywords={},
  pdfsubject={},
  pdfcreator={Emacs 24.5.1 (Org mode 8.2.10)}}
\begin{document}

\maketitle

\section{Chant music}
\label{sec-1}
\subsection{Hildegard von Bingen: O vis aeternatatis, 12$^{\text{th}}$ Century}
\label{sec-1-1}

This chant has a drone and a simple melody, which begins with one singer
and is harmonied by another singer singing the same melody.

\subsection{Soprano, Alto, Tenor, Bass}
\label{sec-1-2}
Soprano is a high female voice. An alto is similarly a female voice.
For male voices, we have the higher tenor and lowest of all the bass.

Using these four voice types, we can create textures in harmony.

\subsection{Tomas Luis de Victoria}
\label{sec-1-3}
Choir that sings the melody at times together and at other times moving apart.

\subsection{Adding in instruments}
\label{sec-1-4}
% Emacs 24.5.1 (Org mode 8.2.10)
\end{document}
